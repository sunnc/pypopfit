\documentclass[a4paper,fleqn,usenatbib,onecolumn]{mnras}
\usepackage{newtxtext,newtxmath}
\usepackage[T1]{fontenc}
\usepackage{ae,aecompl}
\usepackage{graphicx}	% Including figure files
\usepackage{amsmath}	% Advanced maths commands
\let\Bbbk\relax
\usepackage{amssymb}	% Extra maths symbols
\usepackage{mathrsfs}
\usepackage{longtable}
\usepackage{hyperref}
\usepackage[font=small]{caption}
\usepackage{bm}
%\pdfminorversion=5 

\title[pyPopFit]{pyPopFit}

\author[Sun et al.]{Ning-Chen Sun$^1$\thanks{E-mail: n.sun@sheffield.ac.uk}, et al. \\
1 Department of Physics and Astronomy, University of Sheffield, Hicks Building, Hounsfield Road, Sheffield S3 7RH, UK \\}

\date{Accepted XXX. Received YYY; in original form ZZZ}

\pubyear{2022}

\begin{document}
\label{firstpage}
\pagerange{\pageref{firstpage}--\pageref{lastpage}}
\maketitle

%%%%%%%%
\begin{abstract}
Abstract.
\end{abstract}

%%%%%%%%%
\begin{keywords}
supernovae: general
\end{keywords}

%%%%%%%%%%%
\section{Theoretical framework}

\begin{equation}
\mathcal{L}(t^i, A_V^i \mid \bm{D}^i ) \equiv p(\bm{D}^i \mid t^i, A_V^i) = (1-P_{\rm bin}) \times p(\bm{D}^i \mid t^i, A_V^i, {\rm \bf sin}) + P_{\rm bin} \times p(\bm{D}^i \mid t^i, A_V^i, {\rm \bf bin}).
\end{equation}

The single-star part:
\begin{equation}
p(\bm{D}^i \mid t^i, A_V^i, {\rm \bf sin}) = 
\int_{M^\ast_{\rm min}}^{M^\ast_{\rm max}} dM^i \left[ p(\bm{D}^i \mid M^i, t^i, A_V^i, {\rm \bf sin}) p(M^i \mid t^i, {\rm \bf sin}) \right];
\end{equation}

\begin{equation}
p(M^i \mid t^i, {\rm \bf sin}) = 
\begin{cases}
\dfrac{(M^i)^\alpha}{\int_{M^\ast_{\rm min}}^{M_{\rm max}(t^i)} dM^i \left[ (M^i)^\alpha \right]} \equiv \dfrac{(M^i)^\alpha}{A(t^i)} & \text{if $M^\ast_{\rm min}$~$<$~$M^i$~$<$~$M_{\rm max}(t^i)$}, \\
0 & \text{otherwise};
\end{cases}
\end{equation}

\begin{equation}
p(\bm{D}^i \mid t^i, A_V^i, {\rm \bf sin}) =
\int_{M^\ast_{\rm min}}^{M_{\rm max}(t^i)} dM^i \left[ \mathcal{L}_{\rm sin}(M^i, t^i, A_V^i \mid \bm{D}^i) \dfrac{(M^i)^\alpha}{A(t^i)} \right];
\end{equation}

The binary-star part:
\begin{multline}
p(\bm{D}^i \mid t^i, A_V^i, {\rm \bf bin}) = 
\int_{M^\ast_{\rm min}}^{M^\ast_{\rm max}} dM_1^i \int_{M^\ast_{\rm min}}^{M^\ast_{\rm max}} dM_2^i\left[ p(\bm{D}^i \mid M_1^i, M_2^i, t^i, A_V^i, {\rm \bf bin}) p(M_1^i, M_2^i \mid t^i, {\rm \bf bin}) \right] \\
= \int_{M^\ast_{\rm min}}^{M^{\ast}_{\rm max}} dM_1^i \left\{ p(M_1^i) \int_{M^\ast_{\rm min}}^{M^\ast_{\rm max}} dM_2^i\left[ p(\bm{D}^i \mid M_1^i, M_2^i, t^i, A_V^i, {\rm \bf bin}) p(M_2^i \mid  M_1^i, t^i, {\rm \bf bin}) \right] \right\}
\end{multline}

\begin{equation}
p(M_1^i) = \dfrac{(M_1^i)^\alpha}{\int_{M^\ast_{\rm min}}^{M^\ast_{\rm max}} dM^i \left[ (M^i)^\alpha \right]} \equiv \dfrac{(M_1^i)^\alpha}{C} \ \ \ \text{for $M^\ast_{\rm min} < M_1^i < M^\ast_{\rm max}$}
\end{equation}

\begin{equation}
p(M_2^i \mid  M_1^i, t^i, {\rm \bf bin}) =
\begin{cases}
\dfrac{1}{{\rm min} \left[ M_1^i, M_{\rm max}(t^i) \right] - M^\ast_{\rm min}} & \text{if $M_2^i < M_1^i$ and $M_2^i < M_{\rm max}(t^i)$}, \\
0 & \text{otherwise};
\end{cases}
\end{equation}

Divide the binary-star function into two parts
\begin{multline}
p(\bm{D}^i \mid t^i, A_V^i, {\rm \bf bin}) = \int_{M^\ast_{\rm min}}^{M_{\rm max}(t^i)} dM_1^i \left\{ p(M_1^i) \int_{M^\ast_{\rm min}}^{M^\ast_{\rm max}} dM_2^i\left[ p(\bm{D}^i \mid M_1^i, M_2^i, t^i, A_V^i, {\rm \bf bin}) p(M_2^i \mid  M_1^i, t^i, {\rm \bf bin}) \right] \right\} \\
+ \int_{M_{\rm max}(t^i)}^{M^{\ast}_{\rm max}} dM_1^i \left\{ p(M_1^i) \int_{M^\ast_{\rm min}}^{M^\ast_{\rm max}} dM_2^i\left[ p(\bm{D}^i \mid M_1^i, M_2^i, t^i, A_V^i, {\rm \bf bin}) p(M_2^i \mid  M_1^i, t^i, {\rm \bf bin}) \right] \right\} \\
= \int_{M^\ast_{\rm min}}^{M_{\rm max}(t^i)} dM_1^i \left\{ \dfrac{(M_1^i)^\alpha}{C} \int_{M^\ast_{\rm min}}^{M_1^i} dM_2^i\left[ \dfrac{\mathcal{L}_{\rm bin}(M_1^i, M_2^i, t^i, A_V^i \mid \bm{D}^i)}{M_1^i - M^\ast_{\rm min}} \right] \right\} \\
+ \int_{M_{\rm max}(t^i)}^{M^{\ast}_{\rm max}} dM_1^i \left\{ \dfrac{(M_1^i)^\alpha}{C} \int_{M^\ast_{\rm min}}^{M_{\rm max}(t^i)} dM_2^i\left[ \dfrac{\mathcal{L}_{\rm sin}(M_2^i, t^i, A_V^i \mid \bm{D}^i)}{M_{\rm max}(t^i) - M^\ast_{\rm min}} \right] \right\}
\end{multline}

The second term is equal to
\begin{multline}
\int_{M_{\rm max}(t^i)}^{M^{\ast}_{\rm max}} dM_1^i \left[ \dfrac{(M_1^i)^\alpha}{C(M_{\rm max}(t^i) - M^\ast_{\rm min})} \right] \times \int_{M^\ast_{\rm min}}^{M_{\rm max}(t^i)} dM_2^i\left[ \mathcal{L}_{\rm sin}(M_2^i, t^i, A_V^i \mid \bm{D}^i) \right] \\
\equiv \dfrac{B(t^i)}{C(M_{\rm max}(t^i) - M^\ast_{\rm min})} \times \int_{M^\ast_{\rm min}}^{M_{\rm max}(t^i)} dM_2^i\left[ \mathcal{L}_{\rm sin}(M_2^i, t^i, A_V^i \mid \bm{D}^i) \right]
\end{multline}

The total likelihood is
\begin{multline}
p(\bm{D}^i \mid t^i, A_V^i) = 
(1-P_{\rm bin}) \times \int_{M^\ast_{\rm min}}^{M_{\rm max}(t^i)} dM^i \left[ \mathcal{L}_{\rm sin}(M^i, t^i, A_V^i \mid \bm{D}^i) \dfrac{(M^i)^\alpha}{A(t^i)} \right] \\
+ P_{\rm bin} \times \int_{M^\ast_{\rm min}}^{M_{\rm max}(t^i)} dM_1^i \left\{ \dfrac{(M_1^i)^\alpha}{C} \int_{M^\ast_{\rm min}}^{M_1^i} dM_2^i\left[ \dfrac{\mathcal{L}_{\rm bin}(M_1^i, M_2^i, t^i, A_V^i \mid \bm{D}^i)}{M_1^i - M^\ast_{\rm min}} \right] \right\} \\
+ P_{\rm bin} \times \dfrac{B(t^i)}{C(M_{\rm max}(t^i) - M^\ast_{\rm min})} \times \int_{M^\ast_{\rm min}}^{M_{\rm max}(t^i)} dM_2^i\left[ \mathcal{L}_{\rm sin}(M_2^i, t^i, A_V^i \mid \bm{D}^i) \right]
\end{multline}
Changing the label of the integral in the first and third terms do not change their values
\begin{multline}
p(\bm{D}^i \mid t^i, A_V^i)
= \int_{M^\ast_{\rm min}}^{M_{\rm max}(t^i)} dM_1^i \left\{ \left[\dfrac{(1-P_{\rm bin})(M_1^i)^\alpha}{A(t^i)} + \dfrac{P_{\rm bin}B(t^i)}{C (M_{\rm max}(t^i) - M^\ast_{\rm min})} \right] \mathcal{L}_{\rm sin}(M_1^i, t^i, A_V^i \mid \bm{D}^i) \right\} \\
+ \int_{M^\ast_{\rm min}}^{M_{\rm max}(t^i)} dM_1^i \left\{ \dfrac{P_{\rm bin}(M_1^i)^\alpha}{C(M_1^i - M^\ast_{\rm min})} \int_{M^\ast_{\rm min}}^{M_1^i} dM_2^i \left[ \mathcal{L}_{\rm bin}(M_1^i, M_2^i, t^i, A_V^i \mid \bm{D}^i) \right] \right\} \\
\equiv \int_{M^\ast_{\rm min}}^{M_{\rm max}(t^i)} dM_1^i \left[ \mathcal{S}(M_1^i \mid t^i) \mathcal{L}_{\rm sin}(M_1^i, t^i, A_V^i \mid \bm{D}^i ) \right]
+ \int_{M^\ast_{\rm min}}^{M_{\rm max}(t^i)} dM_1^i \left\{ \mathcal{T}(M_1^i) \int_{M^\ast_{\rm min}}^{M_1^i} dM_2^i\left[ \mathcal{L}_{\rm bin}(M_1^i, M_2^i, t^i, A_V^i \mid \bm{D}^i) \right] \right\} \\
\equiv \int_{M^\ast_{\rm min}}^{M_{\rm max}(t^i)} dM_1^i \left\{ \mathcal{S}(M_1^i \mid t^i) \mathcal{L}_{\rm sin}(M_1^i, t^i, A_V^i \mid \bm{D}^i )
+ \mathcal{T}(M_1^i) \int_{M^\ast_{\rm min}}^{M_1^i} dM_2^i\left[ \mathcal{L}_{\rm bin}(M_1^i, M_2^i, t^i, A_V^i \mid \bm{D}^i) \right] \right\}
\end{multline}

The user may specify a value of $M_{\rm min}(t^i)$ below which a star would become too faint to match the observations. Below this value, the likelihood is zero. So the likelihood can be calculated more efficiently with
\begin{multline}
\mathcal{L}(t^i, A_V^i \mid \bm{D}^i ) \equiv  p(\bm{D}^i \mid t^i, A_V^i) = \\
\int_{M_{\rm min}(t^i)}^{M_{\rm max}(t^i)} dM_1^i \left\{ \mathcal{S}(M_1^i \mid t^i) \mathcal{L}_{\rm sin}(M_1^i, t^i, A_V^i \mid \bm{D}^i )
+ \mathcal{T}(M_1^i) \int_{M_{\rm min}(t^i)}^{M_1^i} dM_2^i\left[ \mathcal{L}_{\rm bin}(M_1^i, M_2^i, t^i, A_V^i \mid \bm{D}^i) \right] \right\};
\end{multline}
where
\begin{multline}
\mathcal{S}(M_1^i \mid t^i) = \dfrac{(1-P_{\rm bin})(M_1^i)^\alpha}{A(t^i)} + \dfrac{P_{\rm bin}B(t^i)}{C (M_{\rm max}(t^i) - M^\ast_{\rm min})};
\mathcal{T}(M_1^i) = \dfrac{P_{\rm bin}(M_1^i)^\alpha}{C(M_1^i - M^\ast_{\rm min})}; \\
A(t^i) = \int_{M^\ast_{\rm min}}^{M_{\rm max}(t^i)} dM^i \left[ (M^i)^\alpha \right]; 
B(t^i) = \int_{M_{\rm max}(t^i)}^{M^\ast_{\rm max}} dM^i \left[ (M^i)^\alpha \right]; 
C = \int_{M^\ast_{\rm min}}^{M^\ast_{\rm max}} dM^i \left[ (M^i)^\alpha \right] \equiv A(t^i) + B(t^i).
\end{multline}

%The likelihood function $p(\bm{D}^i \mid M^i, t^i, A_V^i, {\rm \bf sin}) \equiv L^i_{\rm sin}(M^i, t^i, A_V^i)$ can be calculated by comparing the observed magnitudes and errors with the single-star model magnitudes. In summary, the single-star part can be written as
%\begin{equation}
%p(\bm{D}^i \mid t^i, A_V^i, {\rm \bf sin}) = 
%A(t^i) \int_{M^\ast_{\rm min}}^{M_{\rm max}(t^i)} d M^i \left[ L^i_{\rm sin}(M^i, t^i, A_V^i) (M^i)^\alpha \right].
%\end{equation}


%where $M_1^i$ is the primary star's initial mass and $q^i$ is the secondary-to-primary mass ratio. The maximum stellar mass for the primary star in the binary is
%\begin{equation}
%M^{\rm bin}_{\rm 1, max}(q^i, t^i) = {\rm min} \left[ M^{\rm bin}_{\rm 2, max}(t^i)/q, M^\ast_{\rm max}(t^i) \right] = {\rm min} \left[ M^{\rm sin}_{\rm max}(t^i)/q, M^\ast_{\rm max}(t^i) \right] \geq M^{\rm sin}_{\rm max}(t^i);
%\end{equation}
%where $M^\ast_{\rm max}(t^i)$ is the maximum stellar mass (e.g. $\sim300 M_\odot$). A star above $M^{\rm bin}_{\rm 1, max}(q^i, t^i)$ either does not exist or has both stars in the binary that have already exploded. Assuming $q^i$ follows a flat distribution between 0 and 1 which is independent of stellar age, 
%\begin{equation}
%p(q^i \mid t^i, {\rm \bf bin}) = 
%\begin{cases}
%1 & \text{if 0~$<$~$q^i$~$\leq$~1}, \\
%0 & \text{otherwise};
%\end{cases}
%\end{equation}
%$p(M_1^i \mid q^i, t^i, {\rm \bf bin})$ is the IMF of the primary star
%\begin{equation}
%p(M_1^i \mid q^i, t^i, {\rm \bf bin}) =
%\begin{cases}
%B(q^i, t^i)(M_1^i)^\alpha & \text{if $M^\ast_{\rm min}$~$<$~$M_1^i$~$<$~$M^{\rm bin}_{\rm 1, max}(q^i, t^i)$}, \\
%0 & \text{otherwise};
%\end{cases}
%\end{equation}
%where $B(q^i, t^i)$ is the normalization factor
%\begin{equation}
%B(q^i, t^i) = 1.0 \div \int_{M^\ast_{\rm min}}^{M^{\rm bin}_{\rm 1, max}(q^i, t^i)} dM_1^i \left[ (M_1^i)^\alpha \right].
%\end{equation}

%\begin{multline}
%p(\bm{D}^i \mid t^i, A_V^i, {\rm \bf bin}) = 
%\int_{0}^{1} dq^i B(q^i, t^i) \int_{M^\ast_{\rm min}}^{M^{\rm bin}_{\rm 1, max}(q^i, t^i)} dM_1^i \left[ p(\bm{D}^i \mid M_1^i, q^i, t^i, A_V^i, {\rm \bf bin}) (M_1^i)^{-\alpha} \right] \\
%= \int_{0}^{1} dq^i B(q^i, t^i)  \left\{ \int_{M^\ast_{\rm min}}^{M^{\rm sin}_{\rm max}(t^i)} dM_1^i \left[ p(\bm{D}^i \mid M_1^i, q^i, t^i, A_V^i, {\rm \bf bin}) (M_1^i)^{-\alpha} \right] 
%+ \int_{M^{\rm sin}_{\rm max}(t^i)}^{M^{\rm bin}_{\rm 1, max}(q^i, t^i)} dM_1^i \left[ p(\bm{D}^i \mid M_1^i, q^i, t^i, A_V^i, {\rm \bf bin}) (M_1^i)^{-\alpha} \right] \right\};
%\end{multline}
%When both the primary and the secondary stars are still alive, their model magnitude is considered as arising from the simple sum of their fluxes as if both of them are single stars; when, the primary star has already exploded and only the secondary star is alive, the model magnitude for the binary is the same as that for the secondary star.
%\begin{equation}
%p(\bm{D}^i \mid M_1^i, q^i, t^i, A_V^i, {\rm \bf bin}) =
%\begin{cases}
%L^i_{\rm bin}(M_1^i, q^i, t^i, A_V^i) & \text{if $M^\ast_{\rm min} < M_1^i < M^{\rm sin}_{\rm max}$}, \\
%L^i_{\rm sin}(M_2^i, t^i, A_V^i) & \text{if $M^{\rm sin}_{\rm max} < M_1^i < M^{\rm bin}_{\rm 1, max}(q^i, t^i)$};
%\end{cases}
%\end{equation}
%where $M_2^i = q^iM_1^i$ is the secondary star's initial mass.
%\begin{multline}
%p(\bm{D}^i \mid t^i, A_V^i, {\rm \bf bin}) \\
%= \int_{0}^{1} dq^i B(q^i, t^i)  \left\{ \int_{M^\ast_{\rm min}}^{M^{\rm sin}_{\rm max}(t^i)} dM_1^i \left[ L^i_{\rm bin}(M_1^i, q^i, t^i, A_V^i) (M_1^i)^\alpha \right] 
%+ \int_{M^{\rm sin}_{\rm max}(t^i)}^{M^{\rm bin}_{\rm 1, max}(q^i, t^i)} dM_1^i \left[ L^i_{\rm sin}(M_2^i, t^i, A_V^i) (M_1^i)^\alpha \right] \right\} \\
%= \int_{0}^{1} dq^i B(q^i, t^i)  \left\{ \int_{M^\ast_{\rm min}}^{M^{\rm sin}_{\rm max}(t^i)} dM_1^i \left[ L^i_{\rm bin}(M_1^i, q^i, t^i, A_V^i) (M_1^i)^{-\alpha} \right] 
%+ (q^i)^{-\alpha-1}\int_{q^iM^{\rm sin}_{\rm max}(t^i)}^{q^iM^{\rm bin}_{\rm 1, max}(q^i, t^i)} dM_2^i \left[ L^i_{\rm sin}(M_2^i, t^i, A_V^i) (M_2^i)^\alpha \right] \right\}
%\end{multline}


%For the binary-star part
%\begin{equation}
%p(\bm{D}^i \mid t^i, A_V^i, {\rm \bf bin}) = 
%\int_{0}^{1} dq^i \int_{M^\ast_{\rm min}}^{M^{\rm bin}_{\rm 1, max}(q^i, t^i)} dM_1^i \left[ p(\bm{D}^i \mid M_1^i, q^i, t^i, A_V^i, {\rm \bf bin}) p(M_1^i \mid q^i, t^i, {\rm \bf bin}) p(q^i \mid t^i, {\rm \bf bin}) \right] ;
%\end{equation}
%where $M_1^i$ is the primary star's initial mass and $q^i$ is the secondary-to-primary mass ratio. The maximum stellar mass for the primary star in the binary is
%\begin{equation}
%M^{\rm bin}_{\rm 1, max}(q^i, t^i) = {\rm min} \left[ M^{\rm bin}_{\rm 2, max}(t^i)/q, M^\ast_{\rm max}(t^i) \right] = {\rm min} \left[ M^{\rm sin}_{\rm max}(t^i)/q, M^\ast_{\rm max}(t^i) \right] \geq M^{\rm sin}_{\rm max}(t^i);
%\end{equation}
%where $M^\ast_{\rm max}(t^i)$ is the maximum stellar mass (e.g. $\sim300 M_\odot$). A star above $M^{\rm bin}_{\rm 1, max}(q^i, t^i)$ either does not exist or has both stars in the binary that have already exploded. Assuming $q^i$ follows a flat distribution between 0 and 1 which is independent of stellar age, 
%\begin{equation}
%p(q^i \mid t^i, {\rm \bf bin}) = 
%\begin{cases}
%1 & \text{if 0~$<$~$q^i$~$\leq$~1}, \\
%0 & \text{otherwise};
%\end{cases}
%\end{equation}
%$p(M_1^i \mid q^i, t^i, {\rm \bf bin})$ is the IMF of the primary star
%\begin{equation}
%p(M_1^i \mid q^i, t^i, {\rm \bf bin}) =
%\begin{cases}
%B(q^i, t^i)(M_1^i)^\alpha & \text{if $M^\ast_{\rm min}$~$<$~$M_1^i$~$<$~$M^{\rm bin}_{\rm 1, max}(q^i, t^i)$}, \\
%0 & \text{otherwise};
%\end{cases}
%\end{equation}
%where $B(q^i, t^i)$ is the normalization factor
%\begin{equation}
%B(q^i, t^i) = 1.0 \div \int_{M^\ast_{\rm min}}^{M^{\rm bin}_{\rm 1, max}(q^i, t^i)} dM_1^i \left[ (M_1^i)^\alpha \right].
%\end{equation}
%
%%\begin{multline}
%%p(\bm{D}^i \mid t^i, A_V^i, {\rm \bf bin}) = 
%%\int_{0}^{1} dq^i B(q^i, t^i) \int_{M^\ast_{\rm min}}^{M^{\rm bin}_{\rm 1, max}(q^i, t^i)} dM_1^i \left[ p(\bm{D}^i \mid M_1^i, q^i, t^i, A_V^i, {\rm \bf bin}) (M_1^i)^{-\alpha} \right] \\
%%= \int_{0}^{1} dq^i B(q^i, t^i)  \left\{ \int_{M^\ast_{\rm min}}^{M^{\rm sin}_{\rm max}(t^i)} dM_1^i \left[ p(\bm{D}^i \mid M_1^i, q^i, t^i, A_V^i, {\rm \bf bin}) (M_1^i)^{-\alpha} \right] 
%%+ \int_{M^{\rm sin}_{\rm max}(t^i)}^{M^{\rm bin}_{\rm 1, max}(q^i, t^i)} dM_1^i \left[ p(\bm{D}^i \mid M_1^i, q^i, t^i, A_V^i, {\rm \bf bin}) (M_1^i)^{-\alpha} \right] \right\};
%%\end{multline}
%When both the primary and the secondary stars are still alive, their model magnitude is considered as arising from the simple sum of their fluxes as if both of them are single stars; when, the primary star has already exploded and only the secondary star is alive, the model magnitude for the binary is the same as that for the secondary star.
%\begin{equation}
%p(\bm{D}^i \mid M_1^i, q^i, t^i, A_V^i, {\rm \bf bin}) =
%\begin{cases}
%L^i_{\rm bin}(M_1^i, q^i, t^i, A_V^i) & \text{if $M^\ast_{\rm min} < M_1^i < M^{\rm sin}_{\rm max}$}, \\
%L^i_{\rm sin}(M_2^i, t^i, A_V^i) & \text{if $M^{\rm sin}_{\rm max} < M_1^i < M^{\rm bin}_{\rm 1, max}(q^i, t^i)$};
%\end{cases}
%\end{equation}
%where $M_2^i = q^iM_1^i$ is the secondary star's initial mass.
%\begin{multline}
%p(\bm{D}^i \mid t^i, A_V^i, {\rm \bf bin}) \\
%= \int_{0}^{1} dq^i B(q^i, t^i)  \left\{ \int_{M^\ast_{\rm min}}^{M^{\rm sin}_{\rm max}(t^i)} dM_1^i \left[ L^i_{\rm bin}(M_1^i, q^i, t^i, A_V^i) (M_1^i)^\alpha \right] 
%+ \int_{M^{\rm sin}_{\rm max}(t^i)}^{M^{\rm bin}_{\rm 1, max}(q^i, t^i)} dM_1^i \left[ L^i_{\rm sin}(M_2^i, t^i, A_V^i) (M_1^i)^\alpha \right] \right\} \\
%= \int_{0}^{1} dq^i B(q^i, t^i)  \left\{ \int_{M^\ast_{\rm min}}^{M^{\rm sin}_{\rm max}(t^i)} dM_1^i \left[ L^i_{\rm bin}(M_1^i, q^i, t^i, A_V^i) (M_1^i)^{-\alpha} \right] 
%+ (q^i)^{-\alpha-1}\int_{q^iM^{\rm sin}_{\rm max}(t^i)}^{q^iM^{\rm bin}_{\rm 1, max}(q^i, t^i)} dM_2^i \left[ L^i_{\rm sin}(M_2^i, t^i, A_V^i) (M_2^i)^\alpha \right] \right\}
%\end{multline}

%\begin{equation}
%p(\bm{D}^i \mid t^i, A_V^i, {\rm \bf bin}) = 
%\int_{0}^{1} dq^i \int_{M^\ast_{\rm min}}^{M^{\rm bin}_{\rm 1, max}(q^i, t^i)} dM_1^i \left[ p(\bm{D}^i \mid M_1^i, q^i, t^i, A_V^i, {\rm \bf bin}) p(M_1^i \mid q^i, t^i, {\rm \bf bin}) \right] ;
%\end{equation}
%


%\begin{equation}
%p(\bm{D}^i \mid t^i, A_V^i, {\rm \bf bin}) = 
%\int_{M^\ast_{\rm min}}^{M_{\rm max}(q^i, t^i)} p(\bm{D}^i \mid M_1^i, q^i, t^i, A_V^i, {\rm \bf bin}) p(M_1^i \mid q^i, t^i, {\rm \bf bin}) dM_1^i dq^i;
%\end{equation}

%Modelling of the resolved stellar populations is based on \citet{Maund2016} with minor adaption. For a general single star $i$, its model magnitude in the $j$th band ($m^j_{\rm mod}$) can be predicted with its stellar parameters (stellar mass $M^i$, age $t^i$, and extinction $A_V^i$) based on \textsc{parsec} isochrones \citep{parsec.ref}. We calculate its likelihood function by comparing the model magnitudes with its observed photometric data ($\bm{D}^i$):
%\begin{equation}
%\label{cf1.eq}
%p(\bm{D}^i \mid M^i, t^i, A_V^i) = \prod_j  p^i_j,
%\end{equation}
%where $p^i_j$ takes the form:
%\begin{equation}
%p^i_j = \dfrac{1}{\sqrt{2\pi} \, \sigma^j_{\rm obs}} \times\ {\rm exp} \left[ - \dfrac{1}{2} \left( \dfrac{m^j_{\rm mod} (M^i, t^i, A_V^i) - m^j_{\rm obs} }{\sigma^j_{\rm obs}} \right) ^2 \right],
%\end{equation}
%if the star is detected in the $j$th band with magnitude $m^j_{\rm obs}$ and uncertainty $\sigma^j_{\rm obs}$. In case the star is not detected, $p^i_j$ is substituted with
%\begin{equation}
%p^i_j = \dfrac{1}{2} \left[ 1 + {\rm erf} \left( \dfrac{m^j_{\rm mod} (M^i, t^i, A_V^i) -  m^j_{\rm lim}}{\sqrt{2} \, \sigma^j_{\rm lim}} \right) \right],
%\end{equation}
%where $m^j_{\rm lim}$ and $\sigma^j_{\rm lim}$ are the detection limit and its uncertainty for the $j$th band (estimated via artificial star tests; Section~\ref{phot.sec}). Since we are not interested in stellar mass, $M^i$ is treated as a nuisance parameter and can be marginalised over:
%\begin{equation}
%p(\bm{D}^i \mid t^i, A_V^i) = \int \left[ p(\bm{D}^i \mid M^i, t^i, A_V^i) \times p(M^i \mid t^i) \right] dM^i.
%\end{equation}
%We use the \citet{imf.ref} IMF as a prior, thus
%\begin{equation}
%p(M^i \mid t^i) \propto 
%\begin{cases}
%(M^i)^{-2.35} & \text{if $M^i_{\rm min}(t^i)$~$<$~$M^i$~$<$~$M^i_{\rm max}(t^i)$}, \\
%0 & \text{otherwise};
%\end{cases}
%\end{equation}
%where $M^i_{\rm min}(t^i)$ and $M^i_{\rm max}(t^i)$ are the minimum and maximum stellar mass of the isochrone for the age. The likelihood function should meet the normalisation requirement:
%\begin{equation}
%\int p(\bm{D}^i \mid t^i, A_V^i) \,dt^i \,dA_V^i = 1.
%\end{equation}
%
%The \textsc{parsec} isochrones are only for single stars. For a non-interacting binary, the total flux can be considered as the simple sum of fluxes of both stars. In this case, the model magnitudes can still be predicted, which depends on an additional parameter of the secondary-to-primary mass ratio ($q_i$). The likelihood function can be calculated similar to that for single stars; with a flat probability distribution between 0 and 1, $q_i$ is regarded as a nuisance parameter and can be marginalised over.
%
%If we ignore interacting binaries and higher-order multiple systems, a detected source could be either a single star or a non-interacting binary. Assuming a binary fraction of $P_{\rm bin}$~=~0.5, its final likelihood function is the sum of these two scenarios:
%\begin{multline}
%\label{cf2.eq}
%p(\bm{D}^i \mid t^i, A_V^i) = P_{\rm bin} \times p_{\rm bin}(\bm{D}^i \mid t^i, A_V^i) \\
%+ (1 - P_{\rm bin}) \times p_{\rm sin}(\bm{D}^i \mid t^i, A_V^i).
%\end{multline}
%
%The observed stars can be considered as a mixture of model stellar populations, each characterised by an (mean) age and an (mean) extinction ($t^\prime_k$ and $A_V^{k \prime}$ for the $k$th population). In the Bayesian context, this forms a two-level hierarchical process: ($\underline{t^\prime}$, $\underline{A_V^\prime}$) are ``hyper-parameters" that govern the distribution of individual stars' parameters ($\underline{t}$, $\underline{A_V}$), which in turn are related to the photometric data $\underline{\bm{D}}$ via the likelihood function (the underlined quantities are vectors of parameters for all stars or stellar populations). Following the Bayes' theorem, the posterior probability distribution is 
%\begin{multline}
%p(\underline{t^\prime}, \underline{A_V^\prime}, \underline{t}, \underline{A_V} \mid \underline{\bm{D}}) \propto p(\underline{\bm{D}} \mid \underline{t}, \underline{A_V}) \\
%\times p(\underline{t}, \underline{A_V} \mid \underline{t^\prime}, \underline{A_V^\prime}) \times p(\underline{t^\prime}, \underline{A_V^\prime});
%\end{multline}
%the three terms in the right side of the equation are the likelihood, the prior for individual stars' parameters, and the ``hyper-prior" for hyper-parameters of the model stellar populations. Since we are only interested in the model populations, the parameters for individual stars can be marginalised over
%\begin{equation}
%p(\underline{t^\prime}, \underline{A_V^\prime} \mid \underline{\bm{D}}) = \int p(\underline{t^\prime}, \underline{A_V^\prime}, \underline{t}, \underline{A_V} \mid \underline{\bm{D}}) \, d\underline{t} \, d\underline{A_V}.
%\end{equation}
%If we define $w_k$ as the probability of a detected source as a member of the $k$th population ($w_k$ can be treated as new hyper-parameters and fitted numerically), it can be shown that
%\begin{multline}
%p(\underline{t^\prime}, \underline{A_V^\prime} \mid \underline{\bm{D}}) = \left( \prod_i^{N_{\rm star}} \sum_k^{N_{\rm pop}} w_k p_{i, k} \right) \times p(\underline{t^\prime}, \underline{A_V}^\prime), \\
%p_{i, k} = \int \left[ p(\bm{D}^i \mid t^i, A_V^i) \times p(t^i, A_V^i \mid t^{k \prime}, A_V^{k \prime}) \right] \,dt^i \, dA_V^i.
%\end{multline}
%Note that the likelihood $p(\bm{D}^i \mid t^i, A_V^i)$ can be derived with Equations~\ref{cf1.eq}--\ref{cf2.eq}. With pre-defined prior $p(t^i, A_V^i \mid t^{k \prime}, A_V^{k \prime})$ and hyper-prior $p(\underline{t^\prime}, \underline{A_V}^\prime)$, one can obtain the posterior probability distribution for the hyper-parameters. In this way, one can fit model stellar populations to the observed stars and estimate their ages and extinctions.

\bsp	% typesetting comment
\label{lastpage}
\end{document}

% End of mnras_template.tex
